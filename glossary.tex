\newglossaryentry{axa}{
    name=AXA Konzern AG,
    description={Die AXA Konzern AG~\autocite{axa} ist die deutsche Tochtergesellschaft des französischen Versicherungsunternehmens
    AXA und gehört mit über sieben Millionen Kunden zu den größten Versicherungen Deutschlands}
}

\newglossaryentry{illit}{
    name=functional illiterates,
    description={TODO}
}
\newglossaryentry{webscraper}{
    name=web-scraper,
    description={TODO}
}

\newglossaryentry{generalization}{
    name=Generalization,
    description={TODO}
}

\newglossaryentry{causal-lm}{
    name=causal language model,
    description={TODO}
}

\newglossaryentry{capito}{
    name=Capito,
    description={Capito is a company that offers \gls{ats}-Services.
    Their simplifications align with the levels B1 and A2 of the \gls{cefr}~\autocite{capito}}
}

\newglossaryentry{natural-language}{
    name=Natural language,
    description={TODO}
}

\newacronym{nls}{NLS}{Netzwerk Leichte Sprache}
\newacronym[first={Easy Language (in German: \enquote{Leichte Sprache})(EL)}]{el}{EL}{Easy Language}
\newacronym[first={Plain Language (in German: \enquote{Einfache Sprache})(PL)}]{pl}{PL}{Plain Language}
\newacronym{bgg}{BGG}{Act on Equal Opportunities of Persons with Disabilities (in German: Behindertengleichstellungsgesetz)}
\newacronym{ats}{ATS}{Automatic Text Simplification}
\newacronym{nlp}{NLP}{Natural Language Processing}
\newacronym{cefr}{CEFR}{Common European Framework of Reference for Languages}
\newacronym{seq2seq}{seq2seq}{Sequence-to-Sequence}
\newacronym{apa}{APA}{Austria Presse Agentur}
\newacronym[plural=LMMs,longplural={Large Language Models}]{llm}{LLM}{Large Language Model}
\newacronym[plural=SLMs,longplural={Statistical Language Models}]{slm}{SLM}{Statistical Language Model}
\newacronym[plural=NLMs,longplural={Neural Language Models}]{nlm}{NLM}{Neural Language Model}
\newacronym[plural=DNNs,longplural={Deep Neural Networks}]{dnn}{DNN}{Deep Neural Network}
\newacronym{un}{UN CRPD}{United Nations’ Convention on the Rights of People with Disabilities}
\newacronym{lm}{LM}{Language Modeling}



% Abkürzungen
% Beispiele für Verwendung von Akronymen; standardmäßig ist first als "long (short)" definiert
\newacronym{cnn}{CNN}{Convolutional Neural Network}
\newacronym[first={Kreuzvalidierung (engl. Cross Validation, CV)}]{cv}{CV}{Cross Validation}

% keine Worttrennungen in Abkürzungen
% könnte mit Sicherheit auch automatisiert werden aus der obigen Liste, habe ich nicht ausprobiert
\hyphenation{CNN CV}