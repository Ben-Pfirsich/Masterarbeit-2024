\newglossaryentry{axa}{
    name=AXA Konzern AG,
    description={Die AXA Konzern AG~\autocite{axa} ist die deutsche Tochtergesellschaft des französischen Versicherungsunternehmens
    AXA und gehört mit über sieben Millionen Kunden zu den größten Versicherungen Deutschlands}
}

\newglossaryentry{illit}{
    name=functional illiterates,
    description={TODO}
}

\newacronym{nls}{NLS}{Netzwerk Leichte Sprache}
\newacronym{el}{EL}{Einfache Sprache (in German: \enquote{Leichte Sprache})}



% Abkürzungen
% Beispiele für Verwendung von Akronymen; standardmäßig ist first als "long (short)" definiert
\newacronym{cnn}{CNN}{Convolutional Neural Network}
\newacronym[first={Kreuzvalidierung (engl. Cross Validation, CV)}]{cv}{CV}{Cross Validation}

% keine Worttrennungen in Abkürzungen
% könnte mit Sicherheit auch automatisiert werden aus der obigen Liste, habe ich nicht ausprobiert
\hyphenation{CNN CV}