\newglossaryentry{axa}{
    name=AXA Konzern AG,
    description={Die AXA Konzern AG~\autocite{axa} ist die deutsche Tochtergesellschaft des französischen Versicherungsunternehmens
    AXA und gehört mit über sieben Millionen Kunden zu den größten Versicherungen Deutschlands}
}

\newglossaryentry{illit}{
    name=functional illiterates,
    description={TODO}
}

\newacronym{nls}{NLS}{Netzwerk Leichte Sprache}
\newacronym[first={Easy Language (in German: \enquote{Leichte Sprache})}]{el}{EL}{Easy Language}
\newacronym[first={Plain Language (in German: \enquote{Einfache Sprache})}]{pl}{PL}{Plain Language}
\newacronym{bgg}{BGG}{Act on Equal Opportunities of Persons with Disabilities (in German: Behindertengleichstellungsgesetz)}
\newacronym{ats}{ATS}{Automatic Text Simplification}
\newacronym{nlp}{NLP}{Natural Language Processing}
\newacronym{seq2seq}{seq2seq}{Sequence-to-Sequence}



% Abkürzungen
% Beispiele für Verwendung von Akronymen; standardmäßig ist first als "long (short)" definiert
\newacronym{cnn}{CNN}{Convolutional Neural Network}
\newacronym[first={Kreuzvalidierung (engl. Cross Validation, CV)}]{cv}{CV}{Cross Validation}

% keine Worttrennungen in Abkürzungen
% könnte mit Sicherheit auch automatisiert werden aus der obigen Liste, habe ich nicht ausprobiert
\hyphenation{CNN CV}