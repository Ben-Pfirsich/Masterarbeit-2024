\chapter*{Abstract} % kein Eintrag im Inhaltsverzeichnis

%In diesem Kapitel wird die Arbeit kurz und prägnant in maximal einer Seite zusammengefasst.
%Ein klassisches Abstrakt beinhaltet die nachfolgenden Punkte:
%\begin{enumerate}
%	\setlength{\itemsep}{0pt}
%	\item Die Relevanz des betrachteten Themas erläutern. (allgemein/umfassend)
%	\item Die wichtigsten Herausforderungen aufführen und dadurch die Arbeit motivieren.
%	\item Diese Untersuchungen wurden durchgeführt. (knapp)
%	\item Die erhaltenen Resultate wiedergeben.
%	\item Die daraus folgenden Konsequenzen und eventuelle weitere Schritte diskutieren. (allgemein/umfassend)
%\end{enumerate}
%Wie an dem Inhalt der Klammern in der Auflistung zu erkennen ist, besitzt das Abstrakt eine Sanduhr-Struktur, \ie{} der Mittelteil wird sehr knapp bzw. spezifisch abgehandelt, während der Anfang und das Ende allgemeiner bzw. umfassender erläutert werden.
%
%Akronyme wie \zB{} \gls{cnn}, die bereits im Abstrakt verwendet werden, werden durch den Befehl \enquote{glsresetall} im Root-Dokument für den Hauptteil zurückgesetzt.