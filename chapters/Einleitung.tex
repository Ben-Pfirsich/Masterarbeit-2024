\chapter{Introduction}\label{ch:introduction}


\section{Easy Language, Plain Language and Simple Language}\label{sec:el}


From~\autocite{easyLanguageBook}:
aims to improve communication
"comprehensibility and perceptibility"
boost inclusion

broader target groups: communication impairments, dementia, older people with visual impairments,
functional illiterates and language learners

needs to work for everyone, including "primary target groups"


different forms of simplified language exist

ambiguity of terminology
easy, plain, simple with very similar meanings
%hier vielleicht diagramm zitieren

"leicht" in german can also be understood as "light weight"

Easy Language (in German: “Leichte Sprache”)
addresses target groups
does not find acceptance in wider population
making content accessible for everyone -> easy participation in society

rule-based

different guidelines:
    inclusion europe
    Netzwerk Leichte Sprache -> most commonly used
    Barrierefreie-Informationstechnik-Verordnung (BITV 2.0)

rulebooks published by Duden

Netzwerk Leichte Sprache:
    features that go against standard german
    layout provisions

% much more info in this book



Plain Language (in German: “Einfache Sprache”)
not directly intended for disabled people
explaining domain specific knowledge to less informed people, e.g. medical infos
used for inclusive communication in some countries

does not have all necessary properties
more difficult to understand than easy language

closer to standard language

more people accept it and see it positive

idea existing for very long time -> some efforts in early 20th century

in america: guidelines offed by different government institutions
plain english handbooks

people with impairments are not addressed

no fixed rules
more flexible

in Germany:
before easy language: “Citizen-oriented Language” (“Bürgernahe Sprache”) in 1980s
guidelines published by Federal Office of Administration (Bundesverwaltungsamt), last updated in 2002

today: Plain Language as "Einfache Sprache"
dynamic
modified to fit different situations
no clear rule set



Content reduction:
in some versions of easy language



"making expert communication accessible needs much effort" -> automation

UN Convention on the Rights of People with Disabilities (UN CRPD) -> inclusive communication

difficult texts can alienate people

(Web Content Accessibility Guidelines (WCAG 2.0))


Legal Situation:


From~\autocite{un2008}:
Article 9 -Accessibility: "sinage in easy to read and understand forms", "ensure access to information"

Article 2: plain-language listed under communication

Article 29 - Participation in political and public life



From~\autocite{bgg2018}:
public authorities have to present information in easy language (Section 2, paragraph 11) e.g\@. government departments


From~\autocite{netzwerkLS}:
exists since 2006 -> 2013 established as an official association
based on rules

short sentences
easy words
many line breaks
use of paragraphs and headlines


target groups:
(direktes Zitat aus ~\autocite{easyLanguageBook})
• People with a cognitive disability
• People with the illness dementia.
• People who cannot speak German very well.
• People who cannot read very well.

verification of texts through certified examiners


\section{Language Simplification in Natural Language Processing}\label{sec:langSimp}

