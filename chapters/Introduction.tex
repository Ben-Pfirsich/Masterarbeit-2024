\chapter{Introduction}\label{ch:introduction}

In the digital age written text has become ubiquitous. % (TODO: cite?)
% TODO: Examples for important written text: Websites, Online Messages like E-Mails, Newspapers, Government Authorities (distribute Information)

Today the ability to comprehend text is a necessary prerequisite to actively participate in society. % nowadays
This poses a challenge to people with communication impairments~\autocite{easyLanguageBook}.
In Germany about 12\% of the population has difficulties to understand and write standard german because of reduced literacy~\autocite{schomacker2023data}.
Complicated texts can act as a barrier which prevents these people from taking part in everyday life~\autocite{easyLanguageBook}
% could lead to exclusion,

Several attempts have been made to create a type of written language that is more comprehensible and accessible.
The primary goal of such languages is to improve communication for a broad range of people with limited reading and writing skills.
This group includes people with communication impairments, people with dementia, language learners, \gls{illit} and older people with visual impairments~\autocite{easyLanguageBook}.

In German two simplified language have gained popularity in recent years: \enquote{Easy Language} and \enquote{Plain Language}.

%The simplified language has to be simple enough to be understood by everyone.
%At the same time it should not be impractical for people that are proficient in standard language.

\section{Easy Language}\label{sec:el}

\gls{el} is a type of simplified language that is regulated by a set of rules.
There are multiple guidelines published by different organizations that define those rules.
The most commonly used guideline is distributed by the \enquote{\gls{nls}}~\autocite{netzwerkLS, easyLanguageBook}.

The \gls{nls} was founded in 2006 and became an official association in 2013~\autocite{netzwerkHistory}.
The association aims to popularize and standardize \gls{el} in Germany.
All members are volunteers~\autocite{netzwerkGoals}.
Among others, the \gls{nls} includes the following people in their target group:
\begin{itemize}[noitemsep]
    \item people with learning difficulties
    \item people with the illness dementia
    \item people that are not proficient in German
    \item people with reduced reading abilities
\end{itemize}
The \gls{nls} intends that text written in \gls{el} are verified by certified examiners.
Examiners are trained at the \gls{nls} and are usually people in the target group~\autocite{netzwerkPruef}.


\subsection{Rules and Recommendations}\label{subsec:el-rules}
The rules for \gls{el} fall into the categories \enquote{Words}, \enquote{Numbers and Symbols}, \enquote{Sentences}, \enquote{Content} and \enquote{Presentation and Images}.

Words are supposed to be simple.
Technical and foreign words should be avoided.
If difficult words are unavoidable, they have to be explained.
Once a word was introduced to describe a subject, that word should be used again when the subject is referred to.

\begin{center}
    \colorbox{red!20}{
        \begin{minipage}{0.6\textwidth}
        \fontfamily{pag}
           A car drove past me.\\
           The vehicle was very fast.
        \end{minipage}
    }
    \colorbox{green!20}{
        \begin{minipage}{0.6\textwidth}
            \fontfamily{pag}
            A car drove past me.\\
            The car was very fast.
        \end{minipage}
    }
\end{center}
Shorter words are preferred over longer words.
If long words are necessary they should be split with a hyphen character (e.g.\ \enquote{Bundes-Gleichstellungs-Gesetz} instead of \enquote{Bundesgleichstellungsgesetz}).
Additional rules for words are
\begin{itemize}[noitemsep]
    \item the avoidance of acronyms
    \item the use of active over passive
    \item avoidance of genitive and conjunctive
    \item reduced use of negation
\end{itemize}
Numbers and symbols are another aspect that is addressed by the \gls{nls}-guideline.
Numbers should be written in arabic numerals.
For many people digits are easier to read than the spelled out word (e.g.\ \enquote{5 horses} instead of \enquote{five horses}).
Very high number are replaced by rough estimations (e.g.\ \enquote{Many People} instead of \enquote{14.795 People}).
The same goes for percentages.
If unusual symbols (e.g.\ §) are used they need to be explained.

The structure of sentences has a big impact on readability. % TODO: quelle?
In \gls{el} sentences are supposed to be short.
Every sentence should only include one statement.
After each sentence a new line is started.
Sentences with simple structures like \enquote{subject, verb, object} are preferred.
More complex sentences should be broken up in smaller pieces.
These smaller pieces do not necessarily need to form a complete sentence.
\begin{center}
    \colorbox{red!20}{
        \begin{minipage}{0.6\textwidth}
            \fontfamily{pag}
            Do you want to go swimming or watch a movie?
        \end{minipage}
    }
    \colorbox{green!20}{
        \begin{minipage}{0.6\textwidth}
            \fontfamily{pag}
            Do you want to go swimming? \\
            Or watch a movie?
        \end{minipage}
    }
\end{center}
Sentences with subordinate clauses can be broken up similarly.

Topics should not be distributed across the text.
Related content should be kept together.
References to other text are to be avoided.
In the translation process from standard German to \gls{el} content can be omitted if necessary.
Likewise, additional content can be added to make the text more understandable.

The presentation of text is another aspect that impacts clarity and readability.
The \gls{el}-guideline recommends to use a big font size and big line spacing.
Furthermore, text is to be left-aligned.
Backgrounds are supposed to be in light color while the written text is colored darkly.
Text should be structured in many paragraphs with frequent headlines.
Using bullet points instead of comma separated enumerations improves clarity.
Images can be added to accompany the text~\autocite{netzwerkLS}.

\subsection{Prevalence and Adoption of Easy Language}\label{subsec:el-adop}

\gls{el} contains features that go against standard german.
The unique layout with very short lines makes it easy to recognize texts in \gls{el}.
This helps the target group to find easy texts.
But the strong divergence from standard German has been repeatedly criticised.
In 2015 the Federal State Parliament of Schleswig-Holstein sent out information to all voters in \gls{el}.
This caused outrage and refusal from many people~\autocite{easyLanguageBook}.

While \gls{es} hat yet to find acceptance in the general population, it has already been incorporated into german law.
In 2002 the \gls{bgg} was adopted.
The \gls{bgg} guarantees equal living conditions to people regardless of disabilities.
In 2018 the \gls{bgg} was extended to include clearer instructions for accessibility~\autocite{bggInfo}.
Since then public authorities have to provide information in \gls{el} (section 2, paragraph 11)~\autocite{bgg2018}.
Today \gls{el} can be found on many websites of government departments and municipal institution (e.g.\ the City Cologne: \url{https://www.stadt-koeln.de/artikel/07808/index.html}).



%From~\autocite{bgg2018}:


%Sentences are very short and every sentence ends with a new line.


%From~\autocite{easyLanguageBook}:


%ambiguity of terminology
%easy, plain, simple with very similar meanings
%%hier vielleicht diagramm zitieren
%

%
%    different guidelines:
%        inclusion europe
%        Netzwerk Leichte Sprache -> most commonly used
%        Barrierefreie-Informationstechnik-Verordnung (BITV 2.0)
%
%    rulebooks published by Duden
%
%    Netzwerk Leichte Sprache:
%        features that go against standard german
%        layout provisions

%
%
%    % much more info in this book
%
%
%
%Plain Language (in German: “Einfache Sprache”)
%    not directly intended for disabled people
%    explaining domain specific knowledge to less informed people, e.g. medical infos
%    used for inclusive communication in some countries
%
%    does not have all necessary properties
%    more difficult to understand than easy language
%
%    closer to standard language
%
%    more people accept it and see it positive
%
%    idea existing for very long time -> some efforts in early 20th century
%
%    in america: guidelines offed by different government institutions
%    plain english handbooks
%
%    people with impairments are not addressed
%
%    no fixed rules
%    more flexible
%
%    in Germany:
%    before easy language: “Citizen-oriented Language” (“Bürgernahe Sprache”) in 1980s
%    guidelines published by Federal Office of Administration (Bundesverwaltungsamt), last updated in 2002
%
%    today: Plain Language as "Einfache Sprache"
%    dynamic
%    modified to fit different situations
%    no clear rule set
%
%
%
%Content reduction:
%in some versions of easy language
%
%
%
%"making expert communication accessible needs much effort" -> automation
%
%UN Convention on the Rights of People with Disabilities (UN CRPD) -> inclusive communication
%
%difficult texts can alienate people
%
%(Web Content Accessibility Guidelines (WCAG 2.0))
%
%
%
%
%From~\autocite{un2008}:
%Article 9 -Accessibility: "sinage in easy to read and understand forms", "ensure access to information"
%
%Article 2: plain-language listed under communication
%
%Article 29 - Participation in political and public life
%
%
%
%
%
%
%
%From~\autocite{schomacker2023data}:
%"Easy language is roughly equivalent with level A2 of the Common European Framework of Reference for Languages (CEFR)"

\section{Language Simplification in Natural Language Processing}\label{sec:langSimp}


From~\autocite{schomacker2023data}:
text simplification with machine learning is still very new compared to other translation tasks



early rule-based approach to language simplification~\autocite{suter2016}







