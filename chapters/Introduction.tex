\chapter{Introduction}\label{ch:introduction}

In the digital age written text has become ubiquitous (TODO: cite?).
% TODO: Examples for important written text: Websites, Online Messages like E-Mails, Newspapers, Government Authorities (distribute Information)

Today the ability to comprehend text is a necessary prerequisite to actively participate in society. % nowadays
This poses a challenge to people with communication impairments~\autocite{easyLanguageBook}.
In Germany about 12\% of the population has difficulties to understand and write standard german because of reduced literacy~\autocite{schomacker2023data}.
Complicated texts can act as a barrier which prevents these people from taking part in everyday life~\autocite{easyLanguageBook}
% could lead to exclusion,

Several attempts have been made to create a type of written language that is more comprehensible and accessible.
The primary goal of such languages is to improve communication for a broad range of people with limited reading and writing skills.
This group includes people with communication impairments, people with dementia, language learners, functional illiterates(TODO: Glossar) and older people with visual impairments~\autocite{easyLanguageBook}.

In German two simplified language have gained popularity in recent years: \enquote{Easy Language} and \enquote{Plain Language}.

%The simplified language has to be simple enough to be understood by everyone.
%At the same time it should not be impractical for people that are proficient in standard language.

\section{Easy Language}\label{sec:el}

Easy Language (in German: \enquote{Leichte Sprache}) is a type of simplified language that is regulated by a set of rules.
There are multiple guidelines published by different organizations that define those rules.
The most commonly used guideline is distributed by the \enquote{\gls{nls}}~\autocite{netzwerkLS, easyLanguageBook}.

The \gls{nls} was founded in 2006 and became an official association in 2013~\autocite{netzwerkHistory}.
The association aims to popularize and standardize Easy Language in Germany.
All members are volunteers~\autocite{netzwerkGoals}.
Among others, the \gls{nls} includes the following people in their target group:
\begin{itemize}[noitemsep]
    \item people with learning difficulties
    \item people with the illness dementia
    \item people that are not proficient in German
    \item people with reduced reading abilities
\end{itemize}








\begin{center}
    \colorbox{gray!20}{
        \begin{minipage}{0.9\textwidth}
        \fontfamily{pag}
            Das ist ein Text in leichter Sprache.\\
            Mit unterschiedlichen Regeln.
        \end{minipage}
    }
\end{center}

Easy Language contains features that go against standard german.
Sentences are very short and every sentence ends with a new line.
For
%\section{Easy Language, Plain Language and Simple Language}\label{sec:el}


From~\autocite{easyLanguageBook}:


%ambiguity of terminology
%easy, plain, simple with very similar meanings
%%hier vielleicht diagramm zitieren
%
%"leicht" in german can also be understood as "light weight"
%
%
%Easy Language (in German: “Leichte Sprache”)
%    addresses target groups
%    does not find acceptance in wider population
%    making content accessible for everyone -> easy participation in society
%
%    rule-based
%
%    different guidelines:
%        inclusion europe
%        Netzwerk Leichte Sprache -> most commonly used
%        Barrierefreie-Informationstechnik-Verordnung (BITV 2.0)
%
%    rulebooks published by Duden
%
%    Netzwerk Leichte Sprache:
%        features that go against standard german
%        layout provisions

From~\autocite{netzwerkLS}:

short sentences
easy words
many line breaks
use of paragraphs and headlines


verification of texts through certified examiners
    % much more info in this book



Plain Language (in German: “Einfache Sprache”)
    not directly intended for disabled people
    explaining domain specific knowledge to less informed people, e.g. medical infos
    used for inclusive communication in some countries

    does not have all necessary properties
    more difficult to understand than easy language

    closer to standard language

    more people accept it and see it positive

    idea existing for very long time -> some efforts in early 20th century

    in america: guidelines offed by different government institutions
    plain english handbooks

    people with impairments are not addressed

    no fixed rules
    more flexible

    in Germany:
    before easy language: “Citizen-oriented Language” (“Bürgernahe Sprache”) in 1980s
    guidelines published by Federal Office of Administration (Bundesverwaltungsamt), last updated in 2002

    today: Plain Language as "Einfache Sprache"
    dynamic
    modified to fit different situations
    no clear rule set



Content reduction:
in some versions of easy language



"making expert communication accessible needs much effort" -> automation

UN Convention on the Rights of People with Disabilities (UN CRPD) -> inclusive communication

difficult texts can alienate people

(Web Content Accessibility Guidelines (WCAG 2.0))


Legal Situation:


From~\autocite{un2008}:
Article 9 -Accessibility: "sinage in easy to read and understand forms", "ensure access to information"

Article 2: plain-language listed under communication

Article 29 - Participation in political and public life



From~\autocite{bgg2018}:
public authorities have to present information in easy language (Section 2, paragraph 11) e.g\@. government departments




From~\autocite{schomacker2023data}:
12\% of german population has difficulties with reading standard german texts
"Easy language is roughly equivalent with level A2 of the Common European Framework of Reference for Languages (CEFR)"

\section{Language Simplification in Natural Language Processing}\label{sec:langSimp}


From~\autocite{schomacker2023data}:
text simplification with machine learning is still very new compared to other translation tasks



early rule-based approach to language simplification~\autocite{suter2016}







