% =====================================
% Dies ist eine LaTeX-Vorlage für Masterarbeiten und vergleichbare Abschlussarbeiten bei Professor Bialonski.
% Die Vorlage ist aus einer bei Professor Bialonski durchgeführten Masterarbeit hervorgegangen.
% 
% Die Vorlage enthält einige Beispielinhalte, -tabellen und -abbildungen, an denen die Nutzung der verschiedenen Pakete und allgemeine LaTeX-Kommandos zur Ausarbeitung einer Abschlussarbeit zu sehen sind.
% Diese erheben weder einen Anspruch auf Vollständigkeit noch darauf immer den aktuell als Best Practice angesehenen Standards zu folgen.
% 
% Stellen, an denen Entscheidungen über Konfigurationen, verwendete Pakete etc. getroffen werden müssen sind mit einem TODO gekennzeichnet.
% =====================================

% =====================================
% Tipps:
%   für die Nutzung mit einem Versionsverwaltungssystem sollte mindestens mit jedem neuen Satz eine neue Zeile angefangen werden
%   bei der Nutzung des Babel-Pakets mit ngerman kann mit "= ein Bindestrich eingefügt werden, der es LaTeX erlaubt auch an anderen Stellen als dem Bindestrich einen Zeilenumbruch einzufügen, und mit "~ ein geschützter Bindestrich eingefügt werden, d.h. dieser darf nicht als Zeilenumbruch verwendet werden
%   bei der Verwendung z.B. von TeXstudio kann das LanguageTool eingebunden werden, siehe dazu auch https://paulomarconi.github.io/blog/LanguageTool%2BTeXstudio%2BVSCode/ und https://dev.languagetool.org/http-server
%   viele hilfreiche Tipps findet man hier: https://www.semipol.de/posts/2018/06/latex-best-practices-lessons-learned-from-writing-a-phd-thesis/
% =====================================


% Dokumentklasse
% TODO: Dies sind die wichtigsten Einstellungen, die vor Beginn des Schreibprozesses auf jeden Fall festgelegt werden sollten!
%       ich habe die KOMA-Script Abwandlung der report Klasse genutzt und kann dies auch im Nachhinein empfehlen
%       für allgemeine Informationen und Best Practices bezüglich der Auswahl der Schriftgröße und darauf basierend der Seitenränder empfiehlt es sich das Kapitel 2 der KOMA-Script Dokumentation (https://www.ctan.org/pkg/koma-script) zumindest zum Teil durchzulesen
%       ich finde den zweiseitigen Modus angebracht, dies kann allerdings auch zu weiteren Fragen bezüglich der Seitenränder etc. führen; wird der einseitige Modus genutzt, können eventuell die diversen \cleardoubleoddpage Befehle entfernt werden
%       die Bindungskorrektur wird von KOMA-Script dazu genutzt, den Verlust von Papier bei der Bindung in die Wahl der Seitenränder einzubeziehen, 10-12mm sind hier angebracht; für die reine Betrachtung als PDF wäre eigentlich eine Bindungskorrektur von 0 korrekt, dies würde allerdings zwei unterschiedliche Versionen mit ganz anderen Seitenrändern erfordern, von einer solchen Unterscheidung würde ich also abraten
%       DIV ist der Mechanismus der KOMA-Script Klassen für die Bestimmung der Breite der Seitenränder; ein guter Wert kann über "DIV=calc" berechnet werden und aus der log-Datei ausgelesen werden, anschließend kann man davon ausgehend leichte Änderungen vornehmen; für eine Schriftgröße von 12pt und einer Bindungskorrektur von 12mm ist 10 ein guter Wert; alternativ kann das geometry Paket verwendet werden, siehe dazu auch diverse weitere TODOs
\documentclass[
    12pt, % Schriftgröße
    twoside, % zweiseitiger Modus
    ngerman, % deutsches Dokument
    BCOR=12mm, % Bindungskorrektur
    DIV=11, % Division (Anzahl Spalten/Zeilen pro Seite, bestimmt implizit Margins)
    bibliography=toc, % Literatur zu Inhaltsverzeichnis hinzufügen
    openright
]{scrreprt}
%\usepackage{minted}

% Variablen bezüglich des Titels, Autors und Subjects
\newcommand{\titleDocument}{Optimization of German text simplification models with supervised fine-tuning and alignment methods}
\newcommand{\authorDocument}{Ben Pietsch}
\newcommand{\subjectDocument}{Masterarbeit}
\newcommand{\locationDocument}{Jülich}
\newcommand{\dateDocument}{\today} % Alternativ z.B. 30.~September 2021

% grundsätzliche Informationen zum Dokument
\title{\titleDocument}
\author{\authorDocument}
\date{\dateDocument}

% Packete etc.
\usepackage{thesis}

\usepackage{lipsum}
\usepackage{textcomp}
\usepackage{wasysym} % TODO: nur für Beispieltext in summary.tex genutzt, wird nicht benötigt

%\usepackage[cache=false,outputdir=out]{minted}

% Besondere Trennungen (z.B. bei vereinzelter Nutzung englischer Begriffe ohne Nutzung des multilingualen Supports von babel)
\hyphenation{Kon-fi-denz Kon-fi-denz-wert Mus-kel-ak-ti-vi-tät O-pen-Shift Platt-form-en}

\makeglossaries
\newglossaryentry{axa}{
    name=AXA Konzern AG,
    description={Die AXA Konzern AG~\autocite{axa} ist die deutsche Tochtergesellschaft des französischen Versicherungsunternehmens
    AXA und gehört mit über sieben Millionen Kunden zu den größten Versicherungen Deutschlands}
}

\newglossaryentry{illit}{
    name=functional illiterates,
    description={TODO}
}
\newglossaryentry{webscraper}{
    name=web-scraper,
    description={TODO}
}

\newglossaryentry{generalization}{
    name=Generalization,
    description={TODO}
}

\newglossaryentry{causal-lm}{
    name=causal language model,
    description={TODO}
}

\newglossaryentry{capito}{
    name=Capito,
    description={Capito is a company that offers \gls{ats}-Services.
    Their simplifications align with the levels B1 and A2 of the \gls{cefr}~\autocite{capito}}
}

\newglossaryentry{natural-language}{
    name=Natural language,
    description={TODO}
}

\newacronym{nls}{NLS}{Netzwerk Leichte Sprache}
\newacronym[first={Easy Language (in German: \enquote{Leichte Sprache})(EL)}]{el}{EL}{Easy Language}
\newacronym[first={Plain Language (in German: \enquote{Einfache Sprache})(PL)}]{pl}{PL}{Plain Language}
\newacronym{bgg}{BGG}{Act on Equal Opportunities of Persons with Disabilities (in German: Behindertengleichstellungsgesetz)}
\newacronym{ats}{ATS}{Automatic Text Simplification}
\newacronym{nlp}{NLP}{Natural Language Processing}
\newacronym{cefr}{CEFR}{Common European Framework of Reference for Languages}
\newacronym{seq2seq}{seq2seq}{Sequence-to-Sequence}
\newacronym{apa}{APA}{Austria Presse Agentur}
\newacronym[first={Large Language Models (LLM)}]{llm}{LLM}{Large Language Model}
\newacronym[first={Statistical Language Models (SLM)}]{slm}{SLM}{Statistical Language Model}
\newacronym[first={Neural Language Models (NLM)}]{nlm}{NLM}{Neural Language Model}
\newacronym[first={Deep Neural Networks (DNN)}]{dnn}{DNN}{Deep Neural Network}
\newacronym{un}{UN CRPD}{United Nations’ Convention on the Rights of People with Disabilities}
\newacronym{lm}{LM}{Language Modeling}



% Abkürzungen
% Beispiele für Verwendung von Akronymen; standardmäßig ist first als "long (short)" definiert
\newacronym{cnn}{CNN}{Convolutional Neural Network}
\newacronym[first={Kreuzvalidierung (engl. Cross Validation, CV)}]{cv}{CV}{Cross Validation}

% keine Worttrennungen in Abkürzungen
% könnte mit Sicherheit auch automatisiert werden aus der obigen Liste, habe ich nicht ausprobiert
\hyphenation{CNN CV}

\begin{document}
    % ============ Anfang =============
    % Titelseite
    \pagenumbering{roman}
    \include{chapters/title}
    \cleardoublepage

    % Eidesstattliche Erklärung und Abstrakt
    \begingroup

    % keine Seitenzahl und kein running header
        \thispagestyle{empty}
        \renewcommand*{\chapterpagestyle}{empty}


        \cleardoubleoddpage % Eidesstattliche Erklärung rechts, damit Unterschrift nicht durchdrückt
        \chapter*{Eidesstattliche Erklärung} % kein Eintrag im Inhaltsverzeichnis

Ich versichere hiermit, dass ich die vorliegende Masterarbeit mit dem Thema
\begin{quote}
    \textit{\titleDocument}
\end{quote}
selbstständig verfasst und keine anderen als die angegebenen Quellen und Hilfsmittel benutzt habe, wobei ich alle wörtlichen und sinngemäßen Zitate als solche gekennzeichnet habe.
Die Arbeit wurde bisher keiner anderen Prüfungsbehörde vorgelegt und auch nicht veröffentlicht.

\vspace*{2cm}

\begingroup
\setlength{\parindent}{0pt} % keine Einrückung bei neuen Absätzen in diesem Bereich

\locationDocument, den \dateDocument
\bigskip
\bigskip

% gewünschte Breite der Unterschriftslinie
\newlength{\widthbox}
\settowidth{\widthbox}{\locationDocument, den \dateDocument}

\makebox[\widthbox]{\hrulefill}\\
\authorDocument
\endgroup

        \cleardoubleoddpage % Abstrakt rechts
%        \pagestyle{emtpy}
        \chapter*{Abstract} % kein Eintrag im Inhaltsverzeichnis

%In diesem Kapitel wird die Arbeit kurz und prägnant in maximal einer Seite zusammengefasst.
%Ein klassisches Abstrakt beinhaltet die nachfolgenden Punkte:
%\begin{enumerate}
%	\setlength{\itemsep}{0pt}
%	\item Die Relevanz des betrachteten Themas erläutern. (allgemein/umfassend)
%	\item Die wichtigsten Herausforderungen aufführen und dadurch die Arbeit motivieren.
%	\item Diese Untersuchungen wurden durchgeführt. (knapp)
%	\item Die erhaltenen Resultate wiedergeben.
%	\item Die daraus folgenden Konsequenzen und eventuelle weitere Schritte diskutieren. (allgemein/umfassend)
%\end{enumerate}
%Wie an dem Inhalt der Klammern in der Auflistung zu erkennen ist, besitzt das Abstrakt eine Sanduhr-Struktur, \ie{} der Mittelteil wird sehr knapp bzw. spezifisch abgehandelt, während der Anfang und das Ende allgemeiner bzw. umfassender erläutert werden.
%
%Akronyme wie \zB{} \gls{cnn}, die bereits im Abstrakt verwendet werden, werden durch den Befehl \enquote{glsresetall} im Root-Dokument für den Hauptteil zurückgesetzt.

        \glsresetall % alle bereits genutzten Akronyme wieder zurücksetzen
    \endgroup

    % Inhaltsverzeichnis
    \cleardoubleoddpage % Inhaltsverzeichnis rechts
    \begingroup
        \hypersetup{hidelinks}
        \pagestyle{empty}
        \tableofcontents
        \addtocontents{toc}{\protect\thispagestyle{empty}}
        \listoffigures
        \thispagestyle{empty}
    \endgroup
    % =========== Hauptteil ===========
    \cleardoubleoddpage % Beginn Einleitung rechts
    \pagenumbering{arabic}
    \chapter{Introduction}\label{ch:introduction}

In the digital age written text has become ubiquitous. % (TODO: cite?~\autocite{salar-mohtaj-babak-naderi-2022-overview})
% TODO: Examples for important written text: Websites, Online Messages like E-Mails, Newspapers, Government Authorities (distribute Information)

Today the ability to comprehend text is a necessary prerequisite to actively participate in society. % nowadays
This poses a challenge to people with communication impairments~\autocite{easyLanguageBook}.
In Germany about 12\% of the population has difficulties to understand and write standard german because of reduced literacy~\autocite{schomacker2023data}.
Complicated texts can act as a barrier which prevents these people from taking part in everyday life~\autocite{easyLanguageBook}
% could lead to exclusion,

Several attempts have been made to create a type of written language that is more comprehensible and accessible.
The primary goal of such languages is to improve communication for a broad range of people with limited reading and writing skills.
This group includes people with communication impairments, people with dementia, language learners, \gls{illit} and older people with visual impairments~\autocite{easyLanguageBook}.

In German two simplified language have gained popularity in recent years: \enquote{Easy Language} and \enquote{Plain Language}.

%The simplified language has to be simple enough to be understood by everyone.
%At the same time it should not be impractical for people that are proficient in standard language.


\section{Easy Language}\label{sec:el}

\gls{el} is a type of simplified language that is regulated by a set of rules.
There are multiple guidelines published by different organizations that define those rules.
The most commonly used guideline is distributed by the \enquote{\gls{nls}}~\autocite{netzwerkLS, easyLanguageBook}.

The \gls{nls} was founded in 2006 and became an official association in 2013~\autocite{netzwerkHistory}.
The association aims to popularize and standardize \gls{el} in Germany.
All members are volunteers~\autocite{netzwerkGoals}.
Among others, the \gls{nls} includes the following people in their target group:
\begin{itemize}[noitemsep]
    \item people with learning difficulties
    \item people with the illness dementia
    \item people that are not proficient in German
    \item people with reduced reading abilities
\end{itemize}
The \gls{nls} intends that texts written in \gls{el} are verified by certified examiners.
Examiners are trained at the \gls{nls} and are usually people in the target group~\autocite{netzwerkPruef}.

\subsection{Rules and Recommendations}\label{subsec:el-rules}
The rules for \gls{el} fall into the categories \enquote{Words}, \enquote{Numbers and Symbols}, \enquote{Sentences}, \enquote{Content} and \enquote{Presentation and Images}.

Words are supposed to be simple.
Technical and foreign words should be avoided.
If difficult words are unavoidable, they have to be explained.
Once a word was introduced to describe a subject, that word should be used again when the subject is referred to.

\begin{figure}[htb]
    \begin{center}
        \colorbox{badred!20}{
            \begin{minipage}{0.6\textwidth}
                \fontfamily{pag}
                A car drove past me.\\
                It was very fast.\\
                The vehicle was red.
            \end{minipage}
        }
        \colorbox{goodgreen!20}{
            \begin{minipage}{0.6\textwidth}
                \fontfamily{pag}
                A car drove past me.\\
                It was very fast.\\
                The car was red.
            \end{minipage}
        }
    \end{center}
    \caption[Using the same word to refer to the same subject.]{Using the same word to refer to the same subject (bottom). Using synonyms can be confusing (top).}
    \label{fig:subject_ref}
\end{figure}
Shorter words are preferred over longer words.
If long words are necessary they should be split with a hyphen character (e.g.\ \enquote{Bundes-Gleichstellungs-Gesetz} instead of \enquote{Bundesgleichstellungsgesetz}).
Additional rules for words are
\begin{itemize}[noitemsep]
    \item the avoidance of acronyms
    \item the use of active voice over passive voice
    \item avoidance of genitive and conjunctive
    \item reduced use of negation
\end{itemize}
Numbers and symbols are another aspect that is addressed by the \gls{nls}-guideline.
Numbers should be written in arabic numerals.
For many people digits are easier to read than the spelled out word (e.g.\ \enquote{5 horses} instead of \enquote{five horses}).#
Thus, smaller numbers are to be witten in digits.
Very large numbers are replaced by rough estimations (e.g.\ \enquote{Many People} instead of \enquote{14.795 People}).
The same goes for percentages.
If unusual symbols (e.g.\ §) are used they need to be explained.

The structure of sentences has a big impact on readability. % TODO: quelle?
In \gls{el} sentences are supposed to be short.
Every sentence should only include one statement.
After each sentence a new line is started.
Sentences with simple structures like \enquote{subject, verb, object} are preferred.
More complex sentences should be broken up in smaller pieces.
These smaller pieces do not necessarily need to form a complete sentence.
\begin{figure}[htb]
    \begin{center}
        \colorbox{badred!20}{
            \begin{minipage}{0.6\textwidth}
                \fontfamily{pag}
                Do you want to go swimming or watch a movie?
            \end{minipage}
        }
        \colorbox{goodgreen!20}{
            \begin{minipage}{0.6\textwidth}
                \fontfamily{pag}
                Do you want to go swimming? \\
                Or watch a movie?
            \end{minipage}
        }
    \end{center}
    \caption[Splitting longer sentences in \glsentrylong{el}.]{Splitting longer sentences in \glsentrylong{el}. To comply with the rules of \gls{el} the (top) sentence is split in two. The second part of the (bottom) example does not form a complete sentence on its own.}
    \label{fig:split_sentence}
\end{figure}
%\begin{mybox}{Bad example}
%    Do you want to go swimming? \\
%    Or watch a movie?
%\end{mybox}
Sentences with subordinate clauses can be broken up similarly.

Topics should not be distributed across the text.
Related content should be kept together.
References to other texts are to be avoided.
In the translation process from standard German to \gls{el} content can be omitted if necessary.
Likewise, additional content can be added to make the text more understandable.

The presentation of text is another aspect that impacts clarity and readability.
The \gls{el}-guideline recommends to use a big font size and big line spacing.
Furthermore, text is to be left-aligned.
Backgrounds are supposed to be in light color while the written text is colored darkly.
Text should be structured in many paragraphs with frequent headlines.
Using bullet points instead of comma separated enumerations improves clarity.
Images can be added to accompany the text~\autocite{netzwerkLS}.

\subsection{Prevalence and Adoption of Easy Language}\label{subsec:el-adop}

\glsentrylong{el} contains features that go against standard german.
The unique layout with very short lines makes it easy to recognize texts in \gls{el}.
This helps the target group to find texts written in \gls{el} easily.
But the strong divergence from standard German has been repeatedly criticised.
In 2015 the Federal State Parliament of Schleswig-Holstein passed a law to make elections more accessible.
Information regarding elections was sent out to all voters in \gls{el}.
This sparked outrage in the population and was denounced by multiple politicians.
As a result the passed law was reverted~\autocite{easyLanguageBook}.

While \gls{el} has yet to find acceptance in the general population, it has already been incorporated into german law.
In 2002 the \gls{bgg} was adopted.
The \gls{bgg} guarantees equal living conditions to people regardless of disabilities.
In 2018 the \gls{bgg} was extended to include clearer instructions for accessibility~\autocite{bggInfo}.
Since then public authorities have to provide information in \gls{el} as specified in section 2, paragraph 11~\autocite{bgg2018}.
Today \gls{el} can be found on many websites of government departments and municipal institutions (e.g.\ the City Cologne: \url{https://www.stadt-koeln.de/artikel/07808/index.html}).

\begin{figure}
    \centering
    \colorbox{goodgreen!20}{
        \begin{minipage}{0.6\textwidth}
            Fische sind Tiere. \\
            Sie leben im Wasser. \\
            In Flüssen, im Meer und in Seen. \\
            \\
            Fische atmen durch Kiemen. \\
            Das ist ein Körper-teil. \\
            Dadurch können sie unter Wasser atmen.
        \end{minipage}
    }
    \caption[Text written in \glsentrylong{el}.]{Text written in \gls{el} taken from the online dictionary \enquote{Hurraki} (\url{https://hurraki.de/wiki/Fische}).}
    \label{fig:easy_text}
\end{figure}


\section{Plain Language}\label{sec:pl}
\gls{pl} is a simplified language that is much closer to standard language.
Originally, it was not intended for people with disabilities.
\gls{pl} is often used to explain domain specific texts in expert language to less informed people.
Moreover, it addresses language learners e.g.\ migrants and people that learn German as a second language~\autocite{easyLanguageBook}.

In the English language guidelines for \gls{pl} have existed for a long time.
Some early works on more accessible language were published in the beginning of the 20th century.
From the 1960s on the US government pushed to propagate the use of \gls{pl}.
The goal was to improve communication between citizens and experts from administrative institutions.
Several instruction manuals for \gls{pl} were distributed.
Since 2010 federal agencies in the US are required to offer information in a form that is well understood by the citizens.

In Germany efforts for \gls{pl} have only been made very recently.
In the 1980s \enquote{Bürgernahe Sprache} (in English: \enquote{language that is close to the citizens}) was created to make administrative texts more comprehensible.
\enquote{Bürgernahe Sprache} does not take people with communication impairments and less educated people into account.
Thus, it does not satisfy all expected conditions of a simplified language.
In the 2000s multiple approaches for \gls{pl} with a broader target groups were attempted.
But none of them have been widely accepted as a standard yet.
Typical features of \gls{pl} are
\begin{itemize}[noitemsep]
    \item use of common words
    \item use of short words
    \item avoidance of ambiguous words
    \item precise formulations
    \item short sentences
    \item active voice
    \item clear sentence structure (e.g.\ a maximum of two subordinate clauses)
    \item avoidance of acronyms
\end{itemize}
These directives are similar to some of the rules in the \gls{el}-guideline.
But \gls{pl} does not break any rules of standard German.
That might be a reason why \gls{pl} is generally viewed in a more positive light than \gls{el} by many people. % viewed favorably
Moreover, \gls{pl} can be more flexible and less restrictive as there are no fixed rules.
Writers of \gls{pl} are supposed to factor in the audience of the text and make changes accordingly~\autocite{easyLanguageBook}.

\begin{figure}
    \centering
    \includegraphics[width=\linewidth]{images/easy_languages}
    \caption[Different levels of text complexity and comprehensibility.]{Different levels of text complexity and comprehensibility. The different types of languages overlap in certain aspects~\autocite{easyLanguageBook, selbsterstellt}}
    \label{fig:languages}
\end{figure}


%\section{Terminologie}\label{sec:term}

%ambiguity of terminology

%    more difficult to understand than easy language
%
%
%From~\autocite{un2008}:
%Article 9 -Accessibility: "sinage in easy to read and understand forms", "ensure access to information"
%
%Article 2: plain-language listed under communication
%
%Article 29 - Participation in political and public life

%easy, plain, simple with very similar meanings

%    different guidelines:
%        inclusion europe
%        Netzwerk Leichte Sprache -> most commonly used
%        Barrierefreie-Informationstechnik-Verordnung (BITV 2.0)
%
%    rulebooks published by Duden


%

%From~\autocite{schomacker2023data}:
%"Easy language is roughly equivalent with level A2 of the Common European Framework of Reference for Languages (CEFR)"


\section{Automatic Language Simplification}\label{sec:langSimp}

Writing texts in \gls{pl} or \gls{el} and translating complex texts into simpler forms requires much effort~\autocite{easyLanguageBook}.
Hence, there is a demand for tools that can assist in this process.
In the context of \gls{nlp} the task of computer aided reduction of text complexity is known as \gls{ats}~\autocite{Ansch_tz_2023}.
\gls{ats} is similar to other \gls{nlp}-tasks like summarization~\autocite{rios-etal-2021-new} and language translation~\autocite{aumiller2022klexikon}.
It is a sequence-to-sequence (seq2seq) task~\autocite{Ansch_tz_2023}.
% meaning the input and output spaces consist of sequences of varying length
The goal of \gls{ats} is to create a system that takes a text in standard or expert language as an input and outputs a simplified version of that text.

\gls{ats} is a relatively new machine translation task~\autocite{schomacker2023data}.
Especially simplification of German text has not seen a lot of research~\autocite{Ansch_tz_2023}.

An early rule-based approach for language simplification was presented by~\autocite{suter2016}
From~\autocite{Garain2019}: approach for english simplification with parse-trees

In translation solutions based on Deep Learning are generally superior to rule-based methods.
Though, they require vast amounts of Data.~\autocite{otter2019survey}

% TODO: What is monolingual/parallel data?
% TODO: Deep Learning und Supervised Machine Learning
% TODO: Document level vs Sentence Level

From~\autocite{klaper-etal-2013-building}:
build the first parallel corpus for german text simplification
extended by ~\autocite{battisti-etal-2020-corpus}
known as \enquote{Web-Corpus}~\autocite{ebeling2022}, needs to be requested


From~\autocite{Ansch_tz_2023}:
data for german language simplification is scarce
pretraining causal language models improves ATS performance
using monolingual Easy Language Data to fine-tune a GPT-like models
-> decoder output in style of \gls{el}
training ATS task on pretrained mBart model (replacing decoder with above-mentioned GPT-like Decoders, trainig cross-attention only)
publish web scraper for different website with easy language

From~\autocite{klöser2024german}:
previous work
using ChatGPT to generate standard language from monolingual \gls{el} and \gls{pl} to form synthetic parallel data
show that rule-based evaluations for \gls{el} are limited
evaluation of this seq2seq task is very difficult

back-translation (~\autocite{sennrich-etal-2016-improving})

From~\autocite{aumiller2022klexikon}:
German children’s encyclopedia \enquote{Klexikon} aligned with article from Wikipedia in standard language
exploring the task of simultaneous simplification and summarization


From~\autocite{toborek2023new}:
scrapers for different websites
automatic sentence alignment of parallel documents
corpus of 708 aligned documents

From~\autocite{sauberli-etal-2020-benchmarking}
first neural simplification approach based on transformers (~\autocite{Ansch_tz_2023})
"lack of fluency and content preservation"
parallel Austria Presse Agentur (APA) corpus (simplification into levels B1 and A2 (not strictly \gls{el}))
Training transformer models for sentence level simplification


From~\autocite{battisti-etal-2020-corpus}
publish corpus with parallel and monolingual-only (simplified German) data (consisting of about 6,200 documents)
back-translation (~\autocite{sennrich-etal-2016-improving})

From~\autocite{schomacker2023data}:
text simplification with machine learning is still very new compared to other translation tasks

From~\autocite{rios-etal-2021-new}:
20min Dataset (17 905 article pairs (parallel), plain language strongly summarized)
mBART architecture for document level simplification

From~\autocite{spring-etal-2021-exploring}
system that can output different levels of simplified language based on
Common European Framework of References for Languages (CEFR)(~\autocite{Ansch_tz_2023})
controlling copying behaviour



From~\autocite{naderi2019subjective}
TextComplexityDE: 250 parallel sentences

From~\autocite{deilen2023using}
Using sota model ChatGPT-3.5 to prompt for translations into \gls{el}
Outputs text that are easier to understand but leaving out information and not following the rules of \gls{el}

%From~\autocite{madina2023easytoread} TODO: Survey
%From~\autocite{salar-mohtaj-babak-naderi-2022-overview}: TextComplexity 2022

% TODO: Evaluation

BLEU: ~\autocite{papineni-etal-2002-bleu}
SARI: ~\autocite{xu-etal-2016-optimizing}
    \chapter{Large Language Models}\label{ch:techOverview}

Creating machines with human-like language understanding has been a subject of research since the 1950s.
\gls{natural-language}s are highly complex and pose a difficult challenge to computers.
% ambiguous: ~\autocite{quadarLM2020}
\gls{lm} is an approach to make machines read, write and communicate like humans~\autocite{zhao2023survey}.

The main idea of \gls{lm} is to estimate probability distributions over units of texts, e.g.\ words~\autocite{de2015survey}.
Assuming that the occurrence of a word depends on previous words, i.e.\ the context, the probability of that word being next in a sentence can be modeled with a conditional probability~\autocite{jozefowicz2016exploring}.
\[
    P(w_n | w_1, \dots , w_{n-1})
\]
% TODO: quelle finden vielleicht
The ability to estimate the probability distributions over words makes it possible to predict the next word for a given sequence.
In this way, language models can be applied in many \gls{nlp} tasks like speech recognition, machine translation and text summarization~\autocite{jozefowicz2016exploring}.
By simply predicting the next word, language models can hold human-like conversations, which makes it appear as if they understand natural language.
% How would translation look like?
% TODO: Other approaches like BERT Masked language Modeling
%Other approaches to \gls{lm} mask parts of sentences and use all surrounding language units to predict the missing

There are different techniques to model these probabilities of word sequences.
In the 1990s, statistical language models found widespread use.
% TODO: was sind statistical language models
Recent research has focused on neural language models.
Neural based approaches excel at extracting meaningful representations.

\section{notes}

From ~\autocite{quadarLM2020}:
Markov assumption: distribution of a word depends on fixed length of previous words
statistical: n-gram models: tables of conditional probabilities estimated by counting n-gram occurrences and finding relative frequency (P(you | thank) = count(thank you)/count(thank))

deep neural networks more flexible
extracting features of words and finding vector representation that depend on context
\enquote{embedding}: type of representation for a documents where words with similar meaning have similar representations (in mathematical sense)
Embeddings lie in vector spaces and are represented by real vectors
practical for neural networks

From ~\autocite{Hadi_2023}:
\gls{lm}: predict next word or character for a given sequence of text

\gls{llm}s result of rise of deep learning, availability of huge datasets, powerful computing devices
\gls{llm} usually refers to lm with transformer architecture

neural networks better at capturing long-range depencies in text, and complex features
understanding context
breakthrough for llm: "attention is all you need"

during pre-training: models see diverse texts and learn grammar, facts, reasoning
fine-tuning for more specific task ("narrow dataset")

from ~\autocite{Raiaan2024ARO}:
pre-training on large corpora from the web -> learning complicated patterns and language subtleties
fine-tuning on downstream tasks gives state-of-the-art performance
neural language model in 2010s
"comprehend, produce, forecast human language"
BERT and GPT as miles-stones (attention)
improved performance by scaling up models
pre-training with extensive datasets

from~\autocite{zhao2023survey}:
learning features -> help in various nlp tasks
Pre-trained language models (PLM) -> context-aware word representations (versitlie)


\section{Deep Neural Networks}\label{sec:dnn}

\subsection{Multi Layer Perceptron}\label{subsec:multi-layer-perceptron}

\subsection{Learning from Data (Gradient Descent)}\label{subsec:learning-from-data}
% Mathematical Optimization

\subsubsection{Loss Functions}

\subsubsection{Backpropagation}

\section{Transformers}\label{sec:trans}

\section{Decoder-only Models}\label{sec:decoder}

\subsection{GPT}\label{subsec:gpt}

\subsection{Llama}\label{subsec:llama}

\subsubsection{Leo}

\section{Supervised Fine-Tuning (SFT)}\label{sec:supervised-fine-tuning}

\section{Alignment Methods}\label{sec:alignment-methods}

\subsection{RLHF}\label{subsec:rlhf}
\subsection{PPO}\label{subsec:ppo}
\subsection{DPO}\label{subsec:dpo}

    \chapter{Experiments}\label{ch:exp}


\section{Data Overview}\label{sec:data_ov}

\subsection{Plain Language and Easy Language Data}\label{subsec:plain-language-and-easy-language-data}


\section{Supervised Fine-Tuning}\label{sec:supervised-fine-tuning-exp}

\subsection{Architectures}\label{subsec:architectures}

\subsection{Synthetic Parallel}\label{subsec:synthetic-parallel}

% text is not reorganized -> very linear

\subsection{True Parallel}\label{subsec:true-parallel}

% Best model yet


\section{DPO Approach}\label{sec:dpo-approach}

\subsection{DPO-Data}\label{subsec:dpo-data}

\subsection{Annotation Process}\label{subsec:annotation-process}

\subsection{Problems}\label{subsec:problems}

% Computation Limmitation -> unsloth
% Bad annotation predictions -> Model not ready for DPO?
% Bad prediction on unseen domains


\section{Monolingual Pre-Training}\label{sec:monolingual-pre-training}

% Bad results


\section{Evaluation}\label{sec:evaluation}

% Quantitativ, qualitativ


\subsection{Textcomplexity Regression Model}\label{subsec:textcomplexity-regression-model}

    % ============= Ende ==============
%    \printglossary[title={Glossar und Akronyme}]
    \printglossary[type=\acronymtype]

    \printglossary

    \cleardoubleoddpage % Literaturverzeichnis rechts
    \printbibliography
\end{document}
